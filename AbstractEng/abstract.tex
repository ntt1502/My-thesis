\begin{abstractsEnglish}         
Geographic routing is well suited for large-scale wireless sensor networks (WSNs) since it is nearly stateless. It can achieve a near-optimal routing path in the networks without of holes because of its simplicity and scalability. With the occurrence of holes, however, geographic routing faces the problems of hole diffusion and routing path enlargement. The hole shape could be very complex with many cave regions. Several recent proposals attempt to fix these issues by deploying a special, forbidding area around the hole, which helps to improve the congestion on the hole boundary but still are deficient since they use the perimeter routing scheme to bypass the hole within the forbidden area. In this thesis, we introduce a novel approach which is the first to target and solve the near hole routing problem, while ensuring both two requirements: energy consumption and load balancing. Our simulation experiments show that our scheme strongly outperforms the existing schemes in several performance factors, including route stretch, efficient use of energy and load balancing.
\end{abstractsEnglish}
 