\begin{abstracts}         
Một vấn đề quan trọng trong nghiên cứu xây dựng các thuật toán định tuyến cho mạng cảm biến không dây là giải quyết bài toán định tuyến trong mạng có hố. Các hố này có thể được sinh ra bởi một vài lý do, bao gồm do trong môi trường có các vật cản tự nhiên dẫn đến không thể triển khai, hoặc do các thảm họa tự nhiên gây nên. Do đó hình dạng hố rất ngẫu nhiên, có thể là các hình đa giác lõm phức tạp. Một số thuật toán đã được đưa ra để giải quyết bài toán định tuyến với hố bằng việc định nghĩa một vùng cấm quanh hố, kết hợp với một số yếu tố ngẫu nhiên qua đó giúp mạng cân bằng tải và tiêu thụ năng lượng tốt hơn. Tuy nhiên, đối với bài toán định tuyến gần hố, tức là bài toán tìm đường định tuyến có nguồn hoặc đích nằm trong các vùng cấm, các nghiên cứu này đều chưa đưa ra được một thuật toán định tuyến hiệu quả, mà chủ yếu dựa vào thuật toán định tuyến tham lam. Trong đồ án này, chúng tôi đề xuất một thuật toán định tuyến giải quyết bài toán định tuyến gần hố, sử dụng sơ đồ Voronoi để xây dựng lên các đường định tuyến “thoát hố”. Kết quả thực nghiệm cho thấy giải pháp đề xuất có kết quả tốt hơn các thuật toán định tuyến tham lam và thuật toán BOUNDHOLE trên các phương diện: năng lượng, cân bằng tải, độ dài đường định tuyến. 
\end{abstracts}
 